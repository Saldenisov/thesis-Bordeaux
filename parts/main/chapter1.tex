% !TeX spellcheck = en_GB
\chapter{How to use thesis template}
\label{ch:how-to-use}
%------------------------------------------------------
\epigraph{L'art c'est \textbf{moi} -- la science c'est \textbf{nous}.}{french physiologist, Claude Bernard (1813--1878) \index[cite]{Bernard, C.}}
%======================================================
\startcontents[chapter]
\printcontents[chapter]{}{-1}{\setcounter{tocdepth}{3}}
\newpage


\section{General Description}
This thesis template was done in 2013-2015 according to the rules of Université de Bordeaux. It can be compiled with two different chains:
\begin{enumerate}
	\item xelatex -- beamer -- makeglossaries -- xelatex -- xelatex
	\item lualatex -- beamer -- makeglossaries -- lualatex -- lualatex
\end{enumerate}

I as author recommend to use \textbf{MiKTeX} or \textbf{TeX Live} as a \LaTeX distributions, and from my opinion best editors for general purposes are TeXStudio or TeXMaker.


\subsection{Organization}

Thesis is divided in several part:
\begin{enumerate}
	\item Settings
	\item Front Matter (Title Page, Dedication, etc., contents, list of tables and figures)
	\item Main Matter ( Chapters, Appendixes)
	\item Back Matter (Bibliography, Acronyms)
\end{enumerate}

The thesis is divided in to chapters, but not in parts.

The settings of thesis is mainly kept inside of thesis class \textit{thesis-bordeax.cls}. Settings for bibliography could be found in \textit{settings/Bib.tex}, for fonts \textit{settings/Fonts.tex}.

\subsection{ARARA}
I highly recommend to use \textbf{Arara\footnote{Automation of \LaTeX compilation}}. To add Arara command to TEXMaker this nice \textit{post} could be used: \href{http://tex.stackexchange.com/a/107995/43831}{http://tex.stackexchange.com/a/107995/43831}


\section{Floats}

\subsection{Figures/Images/Drawings}
I recommend to use \textit{TikZ}+\textit{pgfplots} packages to built graphs. All presented graphs in this manual are made using this two packages (see \textit{figures} folder for corresponding \textit{tex} file).\par

There are several ways to insert figures in thesis:

\begin{itemize}
	\item Standard way
	\begin{lstlisting}
	\begin{figure}
	\centering
	%\includegraphics[width=6cm]{Name-of-file}
	\caption{Caption.}
	\label{fig:firstFig}
	\end{figure}
	\end{lstlisting}
	
	\item Using \lstinline|\inputfigure| command:
	
	\lstinline|\inputfigure[scale 0 to 1]{File Name}[Caption][Label][type of file pdf, png]|, example: \lstinline|\FloatBarrier\inputfigure[.5]{Chapter1/streak-cameraTIKZ}[Principle scheme of a conventional streak camera][streak-camera][pdf] \FloatBarrier|
	
	\FloatBarrier
	\inputfigure[.5]{Chapter1/streak-cameraTIKZ}[Principle scheme of a conventional streak camera][streak-camera][pdf]
	\FloatBarrier
	
	
	To get an idea hot \textit{TikZ} package works see \textit{figures/Chapter1/streak-cameraTIKZ.tex}:
	
	
	
	\item Using \lstinline|\inputfiguresH}[scale]{Name1}{Name2}[Caption1][Caprtion2][Label1][Label2][Height1][Height1]| command:
	
	
	
	\begin{lstlisting}
	\FloatBarrier
	\inputfiguresH[1]{Chapter1/RuII-GS-bleaching.pdf}{Chapter1/RuII-anth-3growth.pdf}[\acrshort{TRABS} kinetics of ground state bleaching changes for \textbf{2} at $485$\,nm in \acetonitrile ($\lambda_{exc}=465$\,nm)][\acrshort{TRABS} kinetics of anthracene triplet grow-in for \textbf{2} at $430$\,nm in \acetonitrile ($\lambda_{exc}=465$\,nm)][RuII-GS-bleaching][RuII-anth-3growth][5][5]
	\FloatBarrier
	\end{lstlisting}
	
	
	\FloatBarrier
	\inputfiguresH[1]{Chapter1/RuII-GS-bleaching.pdf}{Chapter1/RuII-anth-3growth.pdf}[\acrshort{TRABS} kinetics of ground state bleaching changes for \textbf{2} at $485$\,nm in \acetonitrile ($\lambda_{exc}=465$\,nm)][\acrshort{TRABS} kinetics of anthracene triplet grow-in for \textbf{2} at $430$\,nm in \acetonitrile ($\lambda_{exc}=465$\,nm)][RuII-GS-bleaching][RuII-anth-3growth][5][5]
	\FloatBarrier
	
	
\end{itemize}


\subsection{Tables}


\begin{lstlisting}
\begin{table}[h!]
\addtocounter{totaltables}{1}
\caption{Characteristic time, distance and energy ranges for chemistry and physics.}
\label{table:Ranges}
\centering
\begin{tabular}{|x{2cm}|>{\centering\arraybackslash}x{2.5cm}|>{\centering\arraybackslash}x{3.7cm}|>{\centering\arraybackslash}x{3.1cm}|}

\hline  
&  Time range, s&  Size range, m&  Energy range, eV\\ 
\hline  
Chemistry&  $\rm 10^{-15} - ^*$& $\rm 10^{-10} -10^{1}$  (11~orders) & $\rm 10^{-4} -10^{0}$ (4~orders) \\

\hline  

Physics&  $\rm 10^{-44} -10^{18}$ ($\rm >50$~orders)& $\rm 10^{-35}-10^{-18} -10^{26}$ ($\rm >40$~orders) & up to $\rm 10^{70}$ \\

\hline 
\end{tabular} 
\end{table}
\end{lstlisting}

\begin{table}[h!]
\addtocounter{totaltables}{1}
\caption{Characteristic time, distance and energy ranges for chemistry and physics.}
\label{table:Ranges}
\centering

\begin{tabular}{|x{2cm}|>{\centering\arraybackslash}x{2.5cm}|>{\centering\arraybackslash}x{3.7cm}|>{\centering\arraybackslash}x{3.1cm}|}

\hline  
&  Time range, s&  Size range, m&  Energy range, eV\\ 
\hline  
Chemistry&  $\rm 10^{-15} - ^*$& $\rm 10^{-10} -10^{1}$  (11~orders) & $\rm 10^{-4} -10^{0}$ (4~orders) \\
 
\hline  

Physics&  $\rm 10^{-44} -10^{18}$ ($\rm >50$~orders)& $\rm 10^{-35}-10^{-18} -10^{26}$ ($\rm >40$~orders) & up to $\rm 10^{70}$ \\
 
\hline 
\end{tabular} 
\end{table}

\begin{table}[h!]
\caption{Photophysical properties of Ru(II) complexes in \acetonitrile.}
\label{tab:Ru-properties}

\begin{ThreePartTable}
\centering
%6 columns
\begin{tabu}{{|[2pt] c|c|c|c|c|c|[2pt]}}

\tabucline[2pt]{-}
complex & $\lambda_{em.\,max}$, nm\tnote{a} & $ \Phi_{air}$\tnote{b}&$\Phi_{degas}$\tnote{c} & $\tau$, $\mu$s\tnote{d} & $ K_{eq}$\tnote{f}\\
\tabucline[2pt]{-}

\textbf{1} &686&$2.2 \times 10^{-3}$&$1.3 \times 10^{-2}$& $2.7\pm0.3$&--\\

\tabucline[1pt]{-}

\textbf{2}& 686&$4 \times 10^{-4}$&$9.5 \times 1.3^{-2}$ & $\rm 75\times 10^{-6}$\tnote{e}; $42\pm2$&$15.2\pm2$\\

\tabucline[2pt]{-}


\end{tabu}
\begin{tablenotes}
\footnotesize
\item[a] Recorded on streak camera and uncorrected.

\item[b]  Luminescence QY in air-equilibrated $\rm CH_3CN$ solution \textit{cf.} $\rm [Ru(bpy)_3]^{2+}$ in $\rm H_2O$ (bpy = 2,2'--bipyridine).

\item[c]Luminescence QY in degassed $\rm CH_3CN$ solution \textit{cf.} $\rm [Ru(bpy)_3]^{2+}$ in $\rm H_2O$.

\item[d] MLCT luminescence lifetime in dilute degassed $\rm CH_3CN$.

\item[e] Determined \textit{via} transient absorption spectroscopy in degassed $\rm CH_3CN$. 

\item[f] Excited-state equilibrium constant. 
\end{tablenotes}


\end{ThreePartTable}

\end{table}


\section{Bibliography, Citations and Author Index}

The bibliography print commands are located in \textit{backmatter.tex} file:

\begin{lstlisting}
\addcontentsline{toc}{chapter}{\bibname}
\begingroup
\setstretch{1}
\setlength\bibitemsep{0pt}
\renewcommand*{\bibfont}{\small}
\sloppy %Line breaking
\printbibliography
\endgroup
\end{lstlisting}

The settings are located in \textit{settings/Bib.tex} file. The library itself is stored in \textit{bibliography.thesis-bib.bib} file:

\begin{lstlisting}
@phdthesis{Ducrot-thesis,
author = {Ducrot, A.},
school = {Universit\'{e} Bordeaux 1},
title = {{Synthèses et études de systèmes supramoléculaires photocommutables : récepteurs à ion et molécules entrelacées}},
year=2012,
}

@article{Haas-1971,
author = {Haas, Y. and Stein, G.},
journal = {J. Phys. Chem.},
number = {24},
title = {{Pathways of radiative and radiationless transitions in europium(III) solutions. Role of solvents and anions}},
volume = {75},
year = {1971}
}
\end{lstlisting}

To cite any item from the library the citation command should be used \lstinline|\cite{label}| for example \lstinline|\cite{Berlman1971}| which gives: \cite{Berlman1971}.

The authors from the article automatically added to \textit{author index} by this code located in \textit{settings/Bib.tex}:

\begin{lstlisting}
\usepackage{imakeidx}
\makeindex[intoc,name=cite,title=Author Index,columns=2]
\indexsetup{level=\chapter*,toclevel=chapter,noclearpage}

%Include authors from bibliography to author index.
\DeclareIndexNameFormat{default}{%
\usebibmacro{index:name}{\index[cite]}{#1}{#3}{#5}{#7}}

\DeclareIndexNameFormat{nondefault}{%
\usebibmacro{index:name}{\nocite[cite]}{#1}{#3}{#5}{#7}}  
\end{lstlisting}



\section{Abbreviations}
The acronyms are kept in the file \textit{Settings/abbr.tex}, which you can find  inside of \textit{Bib.tex} file:

\begin{lstlisting}
%Glossaries
\usepackage[nonumberlist,acronym,toc,nomain,nopostdot,nogroupskip]{glossaries}
\input{Settings/abbr}
\makeglossaries
\end{lstlisting}

To set up new acronym this code should be used \lstinline|\newacronym{CSS}{CSS}{Charge Separated State}|.

There are tree ways to use acronym in the text:
\begin{enumerate}
	\item short \lstinline|\acrshort{CSS}| -- \acrshort{CSS}
	\item long  \lstinline|\acrlong{CSS}| -- \acrlong{CSS}
	\item full \lstinline|\acrfull{CSS}| -- \acrfull{CSS}
\end{enumerate}
  
At the of thesis the full list of acronyms will be printed:
\begin{lstlisting}
%Glossary
\glsaddall
\renewcommand*{\glsgroupskip}{}
\printglossary[title=Abbreviations, toctitle=Abbreviations, type=\acronymtype, style=list]
\end{lstlisting}

\section{Indexes}

In the thesis two indexes are created in \textit{settings/Bib.tex} file:

\begin{lstlisting}
\usepackage{imakeidx}
\makeindex[intoc,name=pms,title=General,columns=2]
\makeindex[intoc,name=cite,title=Author Index,columns=2]
\indexsetup{level=\chapter*,toclevel=chapter,noclearpage}
\end{lstlisting}

The \textit{cite} index is used to store cited authors, \lstinline|\cite{}| command automatically adds author in author index. To add manually this command should be used \lstinline|\index[cite]{Bernard, C.}|


The indexes are printed in the of thesis:

\begin{lstlisting}
%INDEXES
\printindex[cite]
%\printindex[pms]
\end{lstlisting}


\section{References}

The package \textit{hyperref} is used to create references in the thesis. To create a reference this commands are used:
\begin{itemize}
	\item reference to chapter, section, etc. \lstinline|\ref{label}|, for example reference to chapter \textit{How to use thesis template}  \lstinline|\ref{ch:how-to-use}| will give \ref{ch:how-to-use}, or figure....
	\item reference to chapter, section, figure, table etc. page \lstinline|p.~\page ref{ch:how-to-use}| -- p.~\pageref{ch:how-to-use}. 
\end{itemize}


\stopcontents[chapter]